%================================================
%	PACKAGES AND THEMES
%================================================
\documentclass[t,aspectratio=169,xcolor=dvipsnames]{beamer}

\usetheme{SimplePlusAIC}

% \usepackage{hyperref} % Beamer loads this automatically
\usepackage{graphicx} % Allows including images
\usepackage{booktabs} % Allows the use of \toprule, \midrule and  \bottomrule in tables
\usepackage{svg} % Allows using svg figures
\usepackage{tcolorbox} % For colored boxes
\usepackage{tikz}
\usepackage{makecell}
\newcommand*{\defeq}{\stackrel{\text{def}}{=}}
\usepackage{setspace}
\usepackage[T1]{fontenc}
\usepackage{helvet}
\usepackage{textgreek}
\usepackage{amsmath}
\usepackage{bm}
\usepackage{ragged2e}
\usepackage{xfrac}
% \usepackage[nice]{nicefrac}
\usepackage[loose]{units}
\usepackage{braket}
\usepackage{physics}
\usepackage{gensymb}
% \usepackage[cm]{sfmath}
\usepackage{verbatim}
\usepackage{fancyvrb}

%\setbeamertemplate{blocks}[rounded][shadow] % block options

\usepackage[svgnames,table]{xcolor}
\arrayrulecolor{black}
\setlength{\arrayrulewidth}{0.20mm}
\renewcommand{\arraystretch}{1.40}  % stretch tables row size

% Custom commands
\newcommand{\tem}[1]{\textbf{\textcolor{red}{#1}}}

%================================================
%	TITLE PAGE
%================================================
\title[Review Arsitektur]{Review Arsitektur Komputer}
\subtitle{Minggu 1: Materi Prasyarat}

\author[lectura.id/course/os]{lectura.id/course/os}
\institute[ITERA]{Program Studi Teknik Informatika \\ Institut Teknologi Sumatera}

\date{\textcolor{nyublue}{2026}}

%================================================
%	BEGIN DOCUMENT 
%================================================
\begin{document}

%------------------------------------------------
%	TITLE SLIDE
%------------------------------------------------
\begin{frame}[plain]

    \titlepage
    
\end{frame}

%------------------------------------------------
%	OUTLINE SLIDE
%------------------------------------------------
\begin{frame}[plain]

    \frametitle{OUTLINE} 
    \framesubtitle{~}

    \begin{spacing}{1.2}
        \tableofcontents
    \end{spacing}
    
\end{frame}

%------------------------------------------------
%	PERTANYAAN PEMBUKA
%------------------------------------------------
\begin{frame}
    \vfill
    \centering
    {\Huge Pernahkah kamu bertanya,}\\[0.5cm]
    {\Huge \tem{bagaimana komputer menjalankan programmu?}}\\[0.8cm]
    {\large Mari kita telusuri bersama.}
    \vfill
\end{frame}

%------------------------------------------------
%	LEARNING OUTCOMES
%------------------------------------------------
\begin{frame}[t]
    \frametitle{Apa yang Akan Kita Pelajari?}
    \framesubtitle{Topik Hari Ini}

    \begin{block}{Fondasi Hardware}
        Kita akan memahami \tem{fondasi hardware} yang menjadi dasar kerja Sistem Operasi.
    \end{block}
    \vspace{0.1cm}
    
    \small
    \begin{enumerate}
        \item Komponen Dasar Komputer (Prosesor, Memori, Penyimpanan)
        \item Input/Output dan Periferal
        \item Bus: Jalur Komunikasi Data
        \item Arsitektur Von Neumann
        \item Siklus Kerja Prosesor
        \item Hierarki Memori
    \end{enumerate}
\end{frame}

\section{Komponen Dasar Komputer}

%------------------------------------------------
\begin{frame}[plain]
    \begin{center}
        \vspace{3cm}
        \Huge\textbf{Komponen Dasar Komputer}\\[0.3cm]
        \large Mengenal Isi Dalam Komputer
    \end{center}
\end{frame}

%------------------------------------------------
\begin{frame}
    \vfill
    \centering
    {\Huge Komputer itu seperti}\\[0.5cm]
    {\Huge \tem{tubuh manusia}}\\[0.8cm]
    {\large Ada otak, tangan, mata, telinga, dan sistem saraf.}
    \vfill
\end{frame}

%------------------------------------------------
\begin{frame}[t]
    \frametitle{Apa Saja Isi Komputer?}
    \framesubtitle{Lima Komponen Utama}
    
    \begin{block}{Komponen Utama}
        Setiap komputer memiliki lima komponen utama yang \tem{bekerja bersama}.
    \end{block}
    \vspace{0.1cm}
    
    \small
    \begin{enumerate}
        \item \textbf{Prosesor (CPU)}: Otak komputer yang berpikir dan menghitung.
        \item \textbf{Memori (RAM)}: Meja kerja sementara untuk menyimpan data yang sedang digunakan.
        \item \textbf{Penyimpanan (Storage)}: Lemari arsip untuk menyimpan data secara permanen.
        \item \textbf{Input/Output}: Cara komputer berkomunikasi dengan dunia luar.
        \item \textbf{Bus}: Jalan raya yang menghubungkan semua komponen.
    \end{enumerate}
\end{frame}

%------------------------------------------------
\begin{frame}[t]
    \frametitle{Prosesor (CPU)}
    \framesubtitle{Otak Komputer}
    
    \begin{alertblock}{Definisi}
        CPU (Central Processing Unit) adalah \tem{otak} komputer yang melakukan semua perhitungan dan pengambilan keputusan.
    \end{alertblock}
    \vspace{0.1cm}
    
    \small
    Bayangkan CPU seperti koki di dapur restoran yang membaca resep (instruksi) dan memasak makanan (mengolah data).
    
    \vspace{0.2cm}
    CPU terdiri dari beberapa bagian:
    \begin{enumerate}
        \item \textbf{ALU (Arithmetic Logic Unit)}: Melakukan operasi matematika (tambah, kurang, kali, bagi) dan logika (lebih besar, sama dengan, atau).
        \item \textbf{Control Unit}: Mengatur urutan kerja, seperti manajer yang memberi perintah.
        \item \textbf{Register}: Kotak kecil super cepat untuk menyimpan data yang sedang diproses.
    \end{enumerate}
\end{frame}

%------------------------------------------------
\begin{frame}[t]
    \frametitle{Seberapa Cepat Prosesor?}
    \framesubtitle{Kecepatan yang Luar Biasa}
    
    \begin{exampleblock}{Fakta Menarik}
        Prosesor modern bisa melakukan \tem{miliaran} operasi per detik!
    \end{exampleblock}
    \vspace{0.1cm}
    
    \small
    Kecepatan CPU diukur dalam Hertz (Hz):
    \begin{itemize}
        \item 1 Hz = 1 operasi per detik
        \item 1 MHz = 1 juta operasi per detik
        \item 1 GHz = 1 miliar operasi per detik
    \end{itemize}
    
    \vspace{0.2cm}
    Komputer modern biasanya punya CPU dengan kecepatan 2-5 GHz. Artinya, dalam waktu kamu berkedip sekali (sekitar 0.3 detik), CPU sudah melakukan 1 miliar operasi!
\end{frame}

%------------------------------------------------
\begin{frame}[t]
    \frametitle{Memori (RAM)}
    \framesubtitle{Meja Kerja Sementara}
    
    \begin{block}{Definisi}
        RAM (Random Access Memory) adalah tempat \tem{sementara} untuk menyimpan data yang sedang aktif digunakan.
    \end{block}
    \vspace{0.1cm}
    
    \small
    Bayangkan RAM seperti meja kerja. Semakin besar mejanya, semakin banyak buku dan kertas yang bisa diletakkan sekaligus.
    
    \vspace{0.2cm}
    \begin{columns}[t]
        \column{0.48\textwidth}
        \textbf{Kelebihan RAM:}
        \begin{itemize}
            \item Sangat cepat diakses
            \item CPU bisa langsung baca/tulis
        \end{itemize}
        
        \column{0.48\textwidth}
        \textbf{Kekurangan RAM:}
        \begin{itemize}
            \item Data hilang saat listrik mati
            \item Kapasitas terbatas (4GB-64GB)
        \end{itemize}
    \end{columns}
\end{frame}

%------------------------------------------------
\begin{frame}[t]
    \frametitle{Mengapa RAM Penting?}
    \framesubtitle{Dampak ke Performa}
    
    \small
    Saat kamu membuka aplikasi, komputer memindahkan program dari penyimpanan ke RAM. Mengapa? Karena RAM jauh lebih cepat!
    
    \vspace{0.2cm}
    \begin{alertblock}{Analogi}
        Membaca buku dari rak (penyimpanan) butuh waktu lama. Lebih cepat membaca dari meja (RAM) yang sudah terbuka.
    \end{alertblock}
    \vspace{0.1cm}
    
    Apa yang terjadi kalau RAM penuh?
    \begin{enumerate}
        \item Komputer jadi \tem{lambat}
        \item Data harus dipindah bolak-balik ke penyimpanan
        \item Aplikasi bisa ``hang'' atau tidak responsif
    \end{enumerate}
\end{frame}

%------------------------------------------------
\begin{frame}[t]
    \frametitle{Penyimpanan (Storage)}
    \framesubtitle{Lemari Arsip Permanen}
    
    \begin{block}{Definisi}
        Penyimpanan adalah tempat menyimpan data secara \tem{permanen}, meskipun komputer dimatikan.
    \end{block}
    \vspace{0.1cm}
    
    \small
    Ada dua jenis utama:
    
    \begin{columns}[t]
        \column{0.48\textwidth}
        \textbf{HDD (Hard Disk Drive)}
        \begin{itemize}
            \item Menggunakan piringan berputar
            \item Lebih lambat
            \item Lebih murah per GB
            \item Contoh: 1TB = Rp 500.000
        \end{itemize}
        
        \column{0.48\textwidth}
        \textbf{SSD (Solid State Drive)}
        \begin{itemize}
            \item Menggunakan chip elektronik
            \item Jauh lebih cepat (5-10x)
            \item Lebih mahal per GB
            \item Contoh: 1TB = Rp 1.500.000
        \end{itemize}
    \end{columns}
\end{frame}

%------------------------------------------------
\begin{frame}[t]
    \frametitle{Perbandingan Kecepatan}
    \framesubtitle{RAM vs SSD vs HDD}
    
    \small
    \centering
    \begin{tabular}{|l|c|c|}
        \hline
        \textbf{Jenis} & \textbf{Kecepatan Baca} & \textbf{Analogi} \\
        \hline
        RAM & 50.000 MB/detik & Mengobrol langsung \\
        \hline
        SSD NVMe & 3.000-7.000 MB/detik & Telepon \\
        \hline
        SSD SATA & 500 MB/detik & Kirim SMS \\
        \hline
        HDD & 100-200 MB/detik & Kirim surat pos \\
        \hline
    \end{tabular}
    
    \vspace{0.3cm}
    \begin{exampleblock}{Kesimpulan}
        RAM \tem{250x lebih cepat} dari HDD! Itulah mengapa program harus dimuat ke RAM dulu.
    \end{exampleblock}
\end{frame}

\section{Input/Output dan Periferal}

%------------------------------------------------
\begin{frame}[plain]
    \begin{center}
        \vspace{3cm}
        \Huge\textbf{Input/Output}\\[0.3cm]
        \large Cara Komputer Berkomunikasi
    \end{center}
\end{frame}

%------------------------------------------------
\begin{frame}
    \vfill
    \centering
    {\Huge Komputer tanpa I/O}\\[0.5cm]
    {\Huge seperti \tem{manusia tanpa indra}}\\[0.8cm]
    {\large Tidak bisa melihat, mendengar, atau berbicara.}
    \vfill
\end{frame}

%------------------------------------------------
\begin{frame}[t]
    \frametitle{Apa itu Input?}
    \framesubtitle{Cara Komputer Menerima Informasi}
    
    \begin{block}{Definisi}
        Input adalah segala cara untuk \tem{memasukkan} data ke dalam komputer.
    \end{block}
    \vspace{0.1cm}
    
    \small
    Contoh perangkat input:
    \begin{columns}[t]
        \column{0.48\textwidth}
        \begin{enumerate}
            \item \textbf{Keyboard}: Memasukkan teks
            \item \textbf{Mouse/Touchpad}: Menunjuk dan mengklik
            \item \textbf{Kamera}: Merekam gambar/video
        \end{enumerate}
        
        \column{0.48\textwidth}
        \begin{enumerate}
            \setcounter{enumi}{3}
            \item \textbf{Mikrofon}: Merekam suara
            \item \textbf{Scanner}: Memindai dokumen
            \item \textbf{Sensor}: Suhu, gerak, dll.
        \end{enumerate}
    \end{columns}
    
    \vspace{0.2cm}
    Tanpa input, komputer tidak tahu apa yang kamu inginkan!
\end{frame}

%------------------------------------------------
\begin{frame}[t]
    \frametitle{Apa itu Output?}
    \framesubtitle{Cara Komputer Menampilkan Hasil}
    
    \begin{alertblock}{Definisi}
        Output adalah segala cara untuk \tem{mengeluarkan} hasil pemrosesan ke dunia luar.
    \end{alertblock}
    \vspace{0.1cm}
    
    \small
    Contoh perangkat output:
    \begin{columns}[t]
        \column{0.48\textwidth}
        \begin{enumerate}
            \item \textbf{Monitor/Layar}: Menampilkan gambar dan teks
            \item \textbf{Speaker}: Mengeluarkan suara
            \item \textbf{Printer}: Mencetak dokumen
        \end{enumerate}
        
        \column{0.48\textwidth}
        \begin{enumerate}
            \setcounter{enumi}{3}
            \item \textbf{Proyektor}: Menampilkan ke layar besar
            \item \textbf{LED/Lampu}: Indikator status
            \item \textbf{Motor/Aktuator}: Menggerakkan sesuatu
        \end{enumerate}
    \end{columns}
    
    \vspace{0.2cm}
    Tanpa output, kamu tidak bisa melihat hasil kerja komputer!
\end{frame}

%------------------------------------------------
\begin{frame}[t]
    \frametitle{Periferal}
    \framesubtitle{Perangkat Tambahan}
    
    \begin{exampleblock}{Definisi}
        Periferal adalah perangkat \tem{eksternal} yang bisa ditambahkan untuk memperluas kemampuan komputer.
    \end{exampleblock}
    \vspace{0.1cm}
    
    \small
    Periferal bukan bagian inti komputer, tapi sangat berguna. Seperti aksesoris pada kendaraan.
    
    \vspace{0.2cm}
    Contoh periferal:
    \begin{enumerate}
        \item \textbf{Flash Drive / USB}: Menyimpan dan memindahkan data
        \item \textbf{External Hard Disk}: Penyimpanan tambahan
        \item \textbf{Webcam}: Kamera eksternal
        \item \textbf{Gamepad/Joystick}: Untuk bermain game
        \item \textbf{Drawing Tablet}: Untuk menggambar digital
    \end{enumerate}
\end{frame}

%------------------------------------------------
\begin{frame}
    \frametitle{Diskusi Singkat}
    \framesubtitle{Pertanyaan untuk Kamu}
    
    \begin{alertblock}{Pertanyaan}
        Sebutkan 3 perangkat yang \tem{bisa berfungsi sebagai Input sekaligus Output}!
    \end{alertblock}
    \vspace{0.3cm}
    
    \centering
    \small
    Diskusikan dengan teman di sebelahmu selama 2 menit!
    
    \vspace{0.3cm}
    \textit{Petunjuk: Pikirkan perangkat yang bisa menerima dan mengirim data.}
\end{frame}

\section{Bus: Jalur Komunikasi}

%------------------------------------------------
\begin{frame}[plain]
    \begin{center}
        \vspace{3cm}
        \Huge\textbf{Bus}\\[0.3cm]
        \large Jalan Raya Data
    \end{center}
\end{frame}

%------------------------------------------------
\begin{frame}
    \vfill
    \centering
    {\Huge Bus adalah}\\[0.5cm]
    {\Huge \tem{jalan raya} di dalam komputer}\\[0.8cm]
    {\large Tempat semua data ``berkendara'' dari satu komponen ke komponen lain.}
    \vfill
\end{frame}

%------------------------------------------------
\begin{frame}[t]
    \frametitle{Apa itu Bus?}
    \framesubtitle{Sistem Komunikasi Internal}
    
    \begin{block}{Definisi}
        Bus adalah sekumpulan kabel/jalur yang menghubungkan \tem{semua komponen} komputer untuk bertukar data.
    \end{block}
    \vspace{0.1cm}
    
    \small
    Bayangkan bus seperti sistem jalan raya di sebuah kota:
    \begin{itemize}
        \item CPU = Kantor pusat pemerintahan
        \item RAM = Gudang logistik
        \item Storage = Arsip kota
        \item I/O = Gerbang masuk/keluar kota
    \end{itemize}
    
    Semua tempat ini dihubungkan oleh jalan (bus) agar bisa saling mengirim dan menerima barang (data).
\end{frame}

%------------------------------------------------
\begin{frame}[t]
    \frametitle{Tiga Jenis Bus}
    \framesubtitle{Jalur yang Berbeda untuk Tujuan Berbeda}
    
    \small
    \begin{columns}[t]
        \column{0.32\textwidth}
        \centering
        {\Large\textcolor{blue}{Data Bus}}\\[0.2cm]
        Membawa \tem{isi data} yang sebenarnya.\\[0.2cm]
        \textit{Seperti truk yang membawa barang.}
        
        \column{0.32\textwidth}
        \centering
        {\Large\textcolor{orange}{Address Bus}}\\[0.2cm]
        Membawa \tem{alamat tujuan} data.\\[0.2cm]
        \textit{Seperti papan petunjuk alamat.}
        
        \column{0.32\textwidth}
        \centering
        {\Large\textcolor{green}{Control Bus}}\\[0.2cm]
        Membawa \tem{perintah kontrol}.\\[0.2cm]
        \textit{Seperti lampu lalu lintas.}
    \end{columns}
    
    \vspace{0.3cm}
    \begin{exampleblock}{Contoh}
        CPU mau baca data dari RAM: Address Bus kirim alamat, Control Bus kirim perintah ``baca'', Data Bus terima hasilnya.
    \end{exampleblock}
\end{frame}

%------------------------------------------------
\begin{frame}[t]
    \frametitle{Lebar Bus}
    \framesubtitle{Berapa Banyak Data Sekaligus?}
    
    \begin{alertblock}{Konsep Penting}
        Semakin \tem{lebar} bus, semakin banyak data yang bisa dikirim sekaligus.
    \end{alertblock}
    \vspace{0.1cm}
    
    \small
    Lebar bus diukur dalam bit:
    \begin{itemize}
        \item Bus 8-bit = bisa kirim 1 byte sekaligus (komputer jadul)
        \item Bus 32-bit = bisa kirim 4 byte sekaligus
        \item Bus 64-bit = bisa kirim 8 byte sekaligus (komputer modern)
    \end{itemize}
    
    \vspace{0.2cm}
    Bayangkan jalan raya: jalan 2 lajur vs jalan tol 8 lajur. Mana yang bisa menampung lebih banyak kendaraan sekaligus?
\end{frame}

\section{Arsitektur vs Organisasi}

%------------------------------------------------
\begin{frame}[plain]
    \begin{center}
        \vspace{3cm}
        \Huge\textbf{Arsitektur vs Organisasi}\\[0.3cm]
        \large Dua Perspektif Berbeda
    \end{center}
\end{frame}

%------------------------------------------------
\begin{frame}[t]
    \frametitle{Bayangkan Sebuah Rumah...}
    \framesubtitle{Analogi Sederhana}
    
    \begin{exampleblock}{Analogi}
        \tem{Arsitektur} adalah denah rumah yang kamu lihat. \tem{Organisasi} adalah cara tukang membangunnya.
    \end{exampleblock}
    \vspace{0.1cm}
    
    \small
    \begin{columns}[t]
        \column{0.48\textwidth}
        \textbf{Arsitektur (Apa?)}\\
        Jumlah kamar, posisi pintu, luas ruangan -- hal yang \textit{terlihat} oleh penghuni.
        
        \column{0.48\textwidth}
        \textbf{Organisasi (Bagaimana?)}\\
        Jenis bahan bangunan, teknik pemasangan -- hal yang \textit{tidak terlihat} oleh penghuni.
    \end{columns}
\end{frame}

%------------------------------------------------
\begin{frame}[t]
    \frametitle{Dalam Dunia Komputer}
    \framesubtitle{Arsitektur vs Organisasi}
    
    \small
    \begin{columns}[t]
        \column{0.48\textwidth}
        \textbf{Arsitektur}\\
        Hal yang \tem{terlihat} oleh programmer:
        \begin{enumerate}
            \item Jenis instruksi yang tersedia
            \item Berapa bit dalam satu data
            \item Cara berkomunikasi dengan perangkat
        \end{enumerate}
        \textit{Contoh: x86, ARM, RISC-V}
        
        \column{0.48\textwidth}
        \textbf{Organisasi}\\
        Cara fitur tersebut \tem{dibangun} secara hardware:
        \begin{enumerate}
            \item Ukuran memori cache
            \item Kecepatan prosesor
            \item Teknologi pembuatan chip
        \end{enumerate}
        \textit{Contoh: Intel i5 vs i7}
    \end{columns}
\end{frame}

%------------------------------------------------
\begin{frame}
    \frametitle{Diskusi Singkat}
    \framesubtitle{Pertanyaan untuk Kamu}
    
    \begin{alertblock}{Pertanyaan}
        Menurutmu, mengapa \tem{Sistem Operasi} perlu memahami keduanya?
    \end{alertblock}
    \vspace{0.3cm}
    
    \centering
    \small
    Diskusikan dengan teman di sebelahmu selama 2 menit!
\end{frame}

\section{Von Neumann}

%------------------------------------------------
\begin{frame}[plain]
    \begin{center}
        \vspace{3cm}
        \Huge\textbf{Arsitektur Von Neumann}\\[0.3cm]
        \large Konsep Program Tersimpan
    \end{center}
\end{frame}

%------------------------------------------------
\begin{frame}
    \vfill
    \centering
    {\Huge Data dan instruksi}\\[0.5cm]
    {\Huge \tem{disimpan di tempat yang sama}}\\[0.8cm]
    {\large Inilah ide revolusioner Von Neumann.}
    \vfill
\end{frame}

%------------------------------------------------
\begin{frame}[t]
    \frametitle{Siapa John Von Neumann?}
    \framesubtitle{Bapak Komputer Modern}
    
    \begin{block}{Tokoh Penting}
        John Von Neumann adalah matematikawan yang pada tahun 1945 mengusulkan \tem{arsitektur komputer} yang masih digunakan hingga hari ini.
    \end{block}
    \vspace{0.1cm}
    
    \small
    Sebelum Von Neumann:
    \begin{itemize}
        \item Komputer diprogram dengan menyambung kabel secara fisik
        \item Mengubah program = mengubah kabel (butuh waktu berhari-hari!)
    \end{itemize}
    
    \vspace{0.2cm}
    Ide brilian Von Neumann:
    \begin{itemize}
        \item Simpan program di \tem{memori} yang sama dengan data
        \item Mengubah program = mengubah isi memori (cukup hitungan detik!)
    \end{itemize}
\end{frame}

%------------------------------------------------
\begin{frame}[t]
    \frametitle{Komponen Von Neumann}
    \framesubtitle{Empat Bagian Utama}
    
    \begin{exampleblock}{Struktur Dasar}
        Arsitektur Von Neumann terdiri dari empat komponen yang saling terhubung melalui \tem{Bus}.
    \end{exampleblock}
    \vspace{0.1cm}
    
    \small
    \begin{enumerate}
        \item \textbf{CPU (Prosesor)}: Mengeksekusi instruksi
            \begin{itemize}
                \item ALU: Melakukan perhitungan
                \item Control Unit: Mengatur alur kerja
            \end{itemize}
        \item \textbf{Memori}: Menyimpan program DAN data
        \item \textbf{Input}: Menerima data dari luar
        \item \textbf{Output}: Mengirim hasil ke luar
    \end{enumerate}
\end{frame}

%------------------------------------------------
\begin{frame}[t]
    \frametitle{Keunggulan dan Kelemahan}
    \framesubtitle{Arsitektur Von Neumann}
    
    \small
    \begin{columns}[t]
        \column{0.48\textwidth}
        \textbf{Keunggulan:}
        \begin{enumerate}
            \item Mudah mengubah program
            \item Desain hardware sederhana
            \item Fleksibel untuk berbagai tugas
        \end{enumerate}
        
        \column{0.48\textwidth}
        \textbf{Kelemahan:}
        \begin{enumerate}
            \item \tem{Von Neumann Bottleneck}: CPU harus menunggu data dari memori
            \item Instruksi dan data berbagi jalur yang sama
        \end{enumerate}
    \end{columns}
    
    \vspace{0.3cm}
    \begin{alertblock}{Fakta}
        Hampir semua komputer yang kamu gunakan (laptop, HP, tablet) masih menggunakan arsitektur ini!
    \end{alertblock}
\end{frame}

\section{Siklus Instruksi}

%------------------------------------------------
\begin{frame}[plain]
    \begin{center}
        \vspace{3cm}
        \Huge\textbf{Siklus Instruksi}\\[0.3cm]
        \large Cara Prosesor Bekerja
    \end{center}
\end{frame}

%------------------------------------------------
\begin{frame}
    \vfill
    \centering
    {\Huge Prosesor bekerja dalam}\\[0.5cm]
    {\Huge \tem{lingkaran tanpa henti}}\\[0.8cm]
    {\large Ambil $\rightarrow$ Pahami $\rightarrow$ Kerjakan $\rightarrow$ Ulangi}
    \vfill
\end{frame}

%------------------------------------------------
\begin{frame}[t]
    \frametitle{Tiga Langkah Utama}
    \framesubtitle{Siklus Fetch-Decode-Execute}
    
    \small
    \begin{columns}[t]
        \column{0.32\textwidth}
        \centering
        {\Huge\textcolor{blue}{1}}\\[0.2cm]
        \textbf{Ambil (Fetch)}\\
        CPU membaca instruksi dari memori berdasarkan alamat yang ditunjuk oleh Program Counter.
        
        \column{0.32\textwidth}
        \centering
        {\Huge\textcolor{orange}{2}}\\[0.2cm]
        \textbf{Pahami (Decode)}\\
        Control Unit menerjemahkan instruksi. Operasi apa? Data mana yang dibutuhkan?
        
        \column{0.32\textwidth}
        \centering
        {\Huge\textcolor{green}{3}}\\[0.2cm]
        \textbf{Kerjakan (Execute)}\\
        ALU atau unit lain mengerjakan perintahnya. Hasilnya disimpan.
    \end{columns}
\end{frame}

%------------------------------------------------
\begin{frame}[t]
    \frametitle{Contoh Sederhana}
    \framesubtitle{Menghitung 5 + 3}
    
    \begin{exampleblock}{Langkah demi Langkah}
        Prosesor tidak langsung tahu jawabannya. Ia harus mengikuti \tem{langkah demi langkah}.
    \end{exampleblock}
    \vspace{0.1cm}
    
    \small
    \begin{enumerate}
        \item \textbf{Fetch}: Ambil instruksi ``LOAD 5 ke Register A''
        \item \textbf{Decode}: Oh, ini perintah memasukkan angka 5 ke tempat A
        \item \textbf{Execute}: Simpan 5 ke Register A
        \item (Ulangi untuk LOAD 3 ke Register B)
        \item \textbf{Fetch}: Ambil instruksi ``ADD A, B''
        \item \textbf{Decode}: Oh, ini perintah menjumlahkan A dan B
        \item \textbf{Execute}: 5 + 3 = 8, simpan hasilnya
    \end{enumerate}
\end{frame}

%------------------------------------------------
\begin{frame}[t]
    \frametitle{Program Counter (PC)}
    \framesubtitle{Penunjuk Instruksi Berikutnya}
    
    \begin{block}{Definisi}
        Program Counter adalah register khusus yang menyimpan \tem{alamat instruksi berikutnya} yang akan dieksekusi.
    \end{block}
    \vspace{0.1cm}
    
    \small
    Cara kerjanya:
    \begin{enumerate}
        \item CPU baca alamat dari PC (misalnya: alamat 100)
        \item CPU ambil instruksi di alamat 100
        \item PC otomatis naik ke alamat berikutnya (101)
        \item Ulangi terus sampai program selesai
    \end{enumerate}
    
    \vspace{0.2cm}
    Bagaimana kalau ada perintah ``lompat''? PC langsung diubah ke alamat tujuan lompatan!
\end{frame}

\section{Hierarki Memori}

%------------------------------------------------
\begin{frame}[plain]
    \begin{center}
        \vspace{3cm}
        \Huge\textbf{Hierarki Memori}\\[0.3cm]
        \large Cepat vs Besar vs Murah
    \end{center}
\end{frame}

%------------------------------------------------
\begin{frame}
    \vfill
    \centering
    {\Huge Tidak ada memori yang}\\[0.5cm]
    {\Huge \tem{cepat, besar, dan murah} sekaligus}\\[0.8cm]
    {\large Kita harus membuat kompromi.}
    \vfill
\end{frame}

%------------------------------------------------
\begin{frame}[t]
    \frametitle{Piramida Memori}
    \framesubtitle{Semakin ke atas: semakin cepat, semakin kecil, semakin mahal}
    
    \small
    \centering
    \begin{tabular}{|c|c|c|c|}
        \hline
        \textbf{Level} & \textbf{Jenis} & \textbf{Ukuran} & \textbf{Kecepatan} \\
        \hline
        1 & Register & Puluhan byte & $<$1 ns \\
        \hline
        2 & Cache L1 & 32-64 KB & 1-2 ns \\
        \hline
        3 & Cache L2/L3 & 256 KB - 32 MB & 5-20 ns \\
        \hline
        4 & RAM & 4-64 GB & 50-100 ns \\
        \hline
        5 & SSD & 256 GB - 4 TB & 50.000-100.000 ns \\
        \hline
        6 & HDD & 1-20 TB & 5.000.000-10.000.000 ns \\
        \hline
    \end{tabular}
    
    \vspace{0.2cm}
    \textit{ns = nanodetik = 0,000000001 detik}
\end{frame}

%------------------------------------------------
\begin{frame}[t]
    \frametitle{Mengapa Hierarki?}
    \framesubtitle{Strategi Cerdas}
    
    \begin{alertblock}{Ide Utama}
        Simpan data yang \tem{sering diakses} di tempat yang cepat, data yang jarang diakses di tempat yang lambat tapi besar.
    \end{alertblock}
    \vspace{0.1cm}
    
    \small
    Analogi perpustakaan:
    \begin{itemize}
        \item Buku yang sedang dibaca = di meja (Register)
        \item Buku referensi yang sering dipakai = di rak dekat meja (Cache)
        \item Buku koleksi ruangan = di rak ruangan (RAM)
        \item Buku di gudang perpustakaan = (Storage)
    \end{itemize}
    
    \vspace{0.2cm}
    Sistem Operasi bertugas mengatur perpindahan data antar level ini secara \tem{otomatis}!
\end{frame}

%------------------------------------------------
\begin{frame}[t]
    \frametitle{Mengapa Sistem Operasi Peduli?}
    \framesubtitle{Tiga Alasan Utama}
    
    \begin{block}{Alasan Utama}
        Sistem Operasi harus memahami hardware untuk bisa \tem{mengelola sumber daya} dengan baik.
    \end{block}
    \vspace{0.1cm}
    
    \small
    \begin{enumerate}
        \item \textbf{Interupsi}: Perangkat I/O memberi sinyal ke CPU saat ada kejadian penting (misal: keyboard ditekan).
        \item \textbf{Proteksi Memori}: Mencegah satu program merusak data program lain.
        \item \textbf{DMA (Direct Memory Access)}: Hardware khusus untuk transfer data langsung ke RAM tanpa membebani CPU.
    \end{enumerate}
\end{frame}

%------------------------------------------------
\begin{frame}
    \frametitle{Refleksi}
    \framesubtitle{Pertanyaan Penutup}
    
    \begin{exampleblock}{Pertanyaan}
        Setelah mempelajari ini, apa yang \tem{paling menarik} menurutmu?
    \end{exampleblock}
    \vspace{0.3cm}
    
    \small
    \begin{columns}[t]
        \column{0.48\textwidth}
        \begin{itemize}
            \item Komponen dasar komputer?
            \item Cara kerja Bus?
            \item Arsitektur Von Neumann?
        \end{itemize}
        
        \column{0.48\textwidth}
        \begin{itemize}
            \item Siklus kerja prosesor?
            \item Hierarki memori?
            \item Peran Sistem Operasi?
        \end{itemize}
    \end{columns}
\end{frame}

%------------------------------------------------
\begin{frame}[t]
    \frametitle{Ringkasan Hari Ini}
    \framesubtitle{Poin-Poin Penting}
    
    \small
    \begin{enumerate}
        \item Komputer terdiri dari \textbf{CPU, Memori, Storage, I/O}, dan \textbf{Bus}.
        \item \textbf{CPU} adalah otak yang mengeksekusi instruksi; \textbf{RAM} adalah meja kerja sementara.
        \item \textbf{Bus} adalah jalan raya data yang menghubungkan semua komponen.
        \item \textbf{Von Neumann}: Program dan data disimpan di memori yang sama.
        \item \textbf{Siklus CPU}: Fetch $\rightarrow$ Decode $\rightarrow$ Execute $\rightarrow$ Ulangi.
        \item \textbf{Hierarki Memori}: Tidak ada yang cepat, besar, dan murah sekaligus.
    \end{enumerate}
\end{frame}

%------------------------------------------------
\begin{frame}[plain]
    \vfill
    \centering
    {\Huge\textcolor{blue}{Selesai!}}\\[0.5cm]
    {\large Siap untuk materi berikutnya?}
    \vfill
\end{frame}

\end{document}
